\documentclass[
    article,
    hidelinks,
    12pt,
    openright,
    twoside,
    a4paper,
    sumario = tradicional,
    brazil
]{abntex2}

\usepackage{babel}

% --- GEOMETRIA (Margens ABNT) ---
\usepackage[
    a4paper,
    top=3cm,
    bottom=2cm,
    left=3cm,
    right=2cm
]{geometry}

% --- FONTES (Times New Roman) ---
\usepackage[T1]{fontenc}
\usepackage{mathptmx}       % Times New Roman
\usepackage[scaled]{helvet} % Helvetica
\usepackage{courier}        % Courier

% --- FORÇAR TIMES NEW ROMAN NOS TÍTULOS ---
\renewcommand{\ABNTEXchapterfont}{\rmfamily\bfseries}
\renewcommand{\ABNTEXsectionfont}{\rmfamily\bfseries}
\renewcommand{\ABNTEXsubsectionfont}{\rmfamily\bfseries}
\renewcommand{\ABNTEXsubsubsectionfont}{\rmfamily\bfseries}

\usepackage{pdfpages}
\usepackage{minted}
\usepackage[utf8]{inputenc}
\usepackage{amsmath}
\usepackage{graphicx}
\usepackage{hyperref}
\usepackage{float}
\usepackage{soul}
\setlength{\intextsep}{1ex}
\usepackage{caption}
\usepackage{nameref}
\usepackage{ragged2e}
\usepackage{csquotes}
\usepackage[vskip=0pt]{quoting}
\SetBlockEnvironment{quoting}
\usepackage{indentfirst}
\setlength{\parskip}{0.0cm}

% --- AJUSTES DE LISTAS (CAIXA ALTA e TIMES) ---
\addto\captionsbrazil{\renewcommand{\listfigurename}{\textbf{\fontfamily{ptm}\selectfont LISTA DE ILUSTRAÇÕES}}}
\addto\captionsbrazil{\renewcommand{\listtablename}{\textbf{\fontfamily{ptm}\selectfont LISTA DE TABELAS}}}

% --- CITAÇÕES ---
\usepackage[alf, abnt-emphasize=bf, abnt-thesis-year=both, abnt-repeated-author-omit=no, abnt-last-names=abnt, abnt-etal-cite=3, abnt-etal-list=3, abnt-etal-text=it, abnt-and-type=e, abnt-doi=doi, abnt-url-package=none, abnt-verbatim-entry=no]{abntex2cite}

% --- ESTILO DE PÁGINA ---
\makepagestyle{abntpages}
\makeevenhead{abntpages}{}{}{\ABNTEXfontereduzida\thepage}
\makeoddhead{abntpages}{}{}{\ABNTEXfontereduzida\thepage}
\aliaspagestyle{chapter}{empty}
\aliaspagestyle{cleared}{empty}
\aliaspagestyle{part}{empty}

% =============================================================================
% INÍCIO DO DOCUMENTO
% =============================================================================
\begin{document}

% Inicia sem numeração visível
\pagestyle{empty}

% =============================================================================
% 1. CAPA (Manual)
% =============================================================================
\begin{center}
    \textbf{UNIVERSIDADE}\\
    \textbf{INSTITUTO}\\
    \textbf{DEPARTAMENTO}\\
    \vspace{1.5cm}
    \textbf{SIGLA - DISCIPLINA\\Prof(a). Dr(a). NOME}\\
    \vspace{3.5cm}
    {\fontsize{18pt}{22.5pt}\selectfont \textbf{TÍTULO DO TRABALHO}}\\
    \vspace{3.5cm}
    
    \noindent\textbf{DISCENTES:}
    \hspace*{5.8cm}
    \vspace{1cm}
    \begin{tabular}{@{}l l@{}}
        \textbf{NOME DO ALUNO 1} & \textbf{NUSP} \\
        \textbf{NOME DO ALUNO 2} & \textbf{NUSP} \\
        \textbf{NOME DO ALUNO 3} & \textbf{NUSP} \\
    \end{tabular}
    
    \vfill
    {São Carlos\\2025}  
\end{center}
% =============================================================================

% Garante que a folha de rosto seja uma página ÍMPAR (Frente)
\cleardoublepage 

% =============================================================================
% 2. FOLHA DE ROSTO (Manual - Para manter consistência com a Capa)
% =============================================================================
% ABNT: A contagem de páginas começa aqui (Ficha de rosto é pag 1 ou 2), 
% mas o número não aparece.
\begin{center}
    % Repete os nomes
    \textbf{NOME DO ALUNO 1}\\
    \textbf{NOME DO ALUNO 2}\\
    \textbf{NOME DO ALUNO 3}\\
    
    \vspace{3.5cm}

    {\fontsize{18pt}{22.5pt}\selectfont \textbf{TÍTULO DO TRABALHO}}\\
    \vspace{3.5cm}
    \hspace{.45\textwidth}
    \begin{minipage}{.5\textwidth}
        \SingleSpacing
        Trabalho acadêmico apresentado ao Instituto tal, da Universidade tal, como requisito parcial para aprovação na disciplina SIGLA - Nome da Disciplina.
        \vspace{\baselineskip}
        Orientador: Prof. Dr. Nome do Professor.
    \end{minipage}

    \vfill
    {São Carlos\\2026}
\end{center}
\cleardoublepage
% =============================================================================
% =============================================================================
% LISTAS
% =============================================================================
% --- Lista de Ilustrações ---
\pdfbookmark[0]{\listfigurename}{lof}
\listoffigures*
\cleardoublepage

% --- Lista de Tabelas ---
\pdfbookmark[0]{\listtablename}{lot}
\listoftables* % <--- ADICIONADO O ASTERISCO (Correção Importante)
\cleardoublepage

% =============================================================================
% SUMÁRIO
% =============================================================================
\chapter*{\textbf{\fontfamily{ptm}\selectfont SUMÁRIO}}
\addcontentsline{toc}{chapter}{SUMÁRIO}
\justifying

\makeatletter
\renewcommand{\tableofcontents}{%
  \begingroup
  \patchcmd{\@makeschapterhead}
    {\@chapapp\space}
    {}
    {}{}
  \renewcommand{\contentsname}{\textbf{SUMÁRIO}}%
  \@starttoc{toc}%
  \endgroup
}
\makeatother
\tableofcontents
\cleardoublepage
% =============================================================================
% ELEMENTOS TEXTUAIS 
% =============================================================================

% Ativa a numeração nas páginas
\pagestyle{abntpages} 

% Reinicia contagem se necessário ou segue fluxo. 
% O abntex2 conta capa, mas usualmente a introdução cai na pag 3, 5 ou 7.

% Para testar sem os arquivos externos, descomente abaixo:
\section{Introdução}

Escreva a introdução aqui. 
% !TeX root = main.tex
\section{Desenvolvimento}

Escreva o desenvolvimento aqui.
% !TeX root = main.tex
\section{Conclusão}

Escreva a conclusão aqui.
\section{Exemplo de imagem}
\vspace{0.5cm}
\begin{figure}
    \centering
    \includegraphics[width=0.5\linewidth]{USP LOGO.png}
    \caption{Caption}
    \label{fig:enter-label}
\end{figure}

\subsection{TABELA}
    {
    \begin{table}[H]
    \centering
    \begin{tabular}{|c|c|} \hline 
         A&  B\\ \hline 
         VALOR 1& 
    VALOR 2 \\ \hline\end{tabular}
    \caption{TITULO TABELA}
    \label{tab:my_label}
    \end{table}
    
    }
   



% =============================================================================
% ELEMENTOS PÓS-TEXTUAIS
% =============================================================================
\postextual
\bibliography{referencias}

\end{document}