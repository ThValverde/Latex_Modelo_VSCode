\documentclass[
article,
hidelinks,
12pt,
openright,
twoside,
a4paper,
sumario = tradicional,
brazil
]{abntex2}
% Language setting

% Replace `english' with e.g. `spanish' to change the document language
\usepackage {babel}

% Set page size and margins
% Replace `letterpaper' with `a4paper' for UK/EU standard size
\usepackage[
    a4paper,
    top=3cm,
    bottom=2cm,
    left=3cm,
    right=2cm, % ABNT
    % marginparwidth=1.75cm % Optional
]{geometry}

% --- PACOTES DE FONTES ---
\usepackage[T1]{fontenc}
\usepackage{mathptmx}      % Times New Roman
\usepackage[scaled]{helvet} % Helvetica
\usepackage{courier}       % Courier
% --- FORÇAR TIMES NEW ROMAN NOS TÍTULOS ---
\renewcommand{\ABNTEXchapterfont}{\rmfamily\bfseries}
\renewcommand{\ABNTEXsectionfont}{\rmfamily\bfseries}
\renewcommand{\ABNTEXsubsectionfont}{\rmfamily\bfseries}
\renewcommand{\ABNTEXsubsubsectionfont}{\rmfamily\bfseries}

\usepackage{pdfpages}
\usepackage{minted}
\usepackage[utf8]{inputenc}
\usepackage{amsmath}
% Useful packages
\usepackage{graphicx}
\usepackage{hyperref}
\usepackage{float}
\usepackage{soul}
\setlength{\intextsep}{1ex}
\usepackage{caption}
%\captionsetup{skip=5pt}
\usepackage{nameref}
\usepackage{ragged2e}
\usepackage{csquotes}
\usepackage[vskip=0pt]{quoting}
\SetBlockEnvironment{quoting}
\usepackage{indentfirst}
\setlength{\parskip}{0.0cm}
% %=-=-=-=-=-=-=-=-=-=-=-=-=-=-=-=-=-=-=-=-=-=-=-=-=-=-=-=-=-=-=
% ALTERAÇÃO DO COMANDO LISTA DE ILUSTRAÇÕES - CAPSLOCK
\addto\captionsbrazil{\renewcommand{\listfigurename}{\textbf{\fontfamily{ptm}\selectfont LISTA DE ILUSTRAÇÕES}}}
% ALTERAÇÃO DO COMANDO LISTA DE TABELAS
\addto\captionsbrazil{\renewcommand{\listtablename}{\textbf{\fontfamily{ptm}\selectfont LISTA DE TABELAS}}}
\usepackage[alf, abnt-emphasize=bf, abnt-thesis-year=both, abnt-repeated-author-omit=no, abnt-last-names=abnt, abnt-etal-cite=3, abnt-etal-list=3, abnt-etal-text=it, abnt-and-type=e, abnt-doi=doi, abnt-url-package=none, abnt-verbatim-entry=no]{abntex2cite}
%\bibliographystyle{USPSC-classe/abntex2-alf-USPSC}
% =============================================================================
% CONFIGURAÇÃO DE NUMERAÇÃO DE PÁGINAS CONFORME ABNT NBR 14724:2011
% =============================================================================
% Configuração do estilo de página para elementos textuais 
% Numeração no canto superior direito conforme ABNT
\makepagestyle{abntpages}
\makeevenhead{abntpages}{}{}{\ABNTEXfontereduzida\thepage}
\makeoddhead{abntpages}{}{}{\ABNTEXfontereduzida\thepage}
% Remove numeração de páginas especiais (capítulos, etc.)
\aliaspagestyle{chapter}{empty}
\aliaspagestyle{cleared}{empty}
\aliaspagestyle{part}{empty}
% =============================================================================
% Informações de dados para CAPA e FOLHA DE ROSTO
% =============================================================================
\title{TITULO DO PROJETO}%opicional-não aparece no documento
\author{NOME}%opicional-não aparece no documento
% =============================================================================
\begin{document}
% =============================================================================
% ELEMENTOS PRÉ-TEXTUAIS
% Conforme ABNT: contados a partir da folha de rosto, mas sem numeração visível
% =============================================================================
% Inicia contagem de páginas, mas sem mostrar numeração
\pagenumbering{arabic}
\pagestyle{empty}
% CAPA (não conta para numeração segundo ABNT)
\begin{center}
% =============================================================================
%\includegraphics[width=0.75\linewidth]{logo_eesc_horizontal_com_subtitulo_portugues.pdf}\%\includegraphics[width=0.75\linewidth]{logo_eesc_horizontal_com_subtitulo_portugues.pdf}\\ 
%logo faculdade - opcional - ajustar altura em vspace caso não for utilizar (para 4 e 6 - recomendação)
\textbf{UNIVERSIDADE}\\
\textbf{INSTITUTO}\\
\textbf{DEPARTAMENTO}\\
\vspace{1.5cm}
\textbf{SIGLA - DISCIPLINA\\Prof(a). Dr(a). NOME}\\
\textbf{Estagiário/Monitor}\\
\vspace{3.5cm}
{\fontsize{18pt}{22.5pt}\selectfont \textbf{
Título}}\\

\vspace{3.5cm}
\noindent\textbf{DISCENTES:}

\vspace{1cm}

\hspace*{5.8cm} % <-- ajuste aqui para alinhar com o "D" acima
\begin{tabular}{@{}l l@{}}
\textbf{NOME} & \textbf{NUSP} \\
\textbf{NOME} & \textbf{NUSP} \\
\textbf{NOME} & \textbf{NUSP} \\
\textbf{NOME} & \textbf{NUSP} \\
\end{tabular}
\vfill
\begin{center}
{São Carlos\\2025}  
\end{center}
\end{center}

% Fim da CAPA - a partir daqui inicia contagem das páginas
\newpage
\cleardoublepage

% =============================================================================
% LISTAS (ILUSTRAÇÕES E TABELAS)
% =============================================================================
% --- Lista de Ilustrações ---
\pdfbookmark[0]{\listfigurename}{lof} % Adiciona link no menu lateral do PDF
\listoffigures* % Gera a lista
\cleardoublepage
% --- Lista de Tabelas ---
\pdfbookmark[0]{\listtablename}{lot}  % Adiciona link no menu lateral do PDF
\listoftables % Gera a lista
\cleardoublepage
% =============================================================================
% SUMÁRIO
% =============================================================================
\chapter*{\textbf{\fontfamily{ptm}\selectfont SUMÁRIO}}
\addcontentsline{toc}{chapter}{SUMÁRIO}
\justifying

\makeatletter
\renewcommand{\tableofcontents}{%

  \begingroup
  \patchcmd{\@makeschapterhead}
    {\@chapapp\space}
    {}
    {}{}
  \renewcommand{\contentsname}{\textbf{SUMÁRIO}}%
  \@starttoc{toc}%
  \endgroup
}
\makeatother
\tableofcontents
%\cleardoublepage
\newpage
% =============================================================================
% ELEMENTOS TEXTUAIS  
% Conforme ABNT: primeira página com numeração visível (introdução)
% =============================================================================
% Ativa numeração visível no canto superior direito
\pagestyle{abntpages} % comente para tirar a numeração
%=====================================================================
\section{Introdução}

Escreva a introdução aqui.
% !TeX root = main.tex
\section{Desenvolvimento}

Escreva o desenvolvimento aqui.
% !TeX root = main.tex
\section{Conclusão}

Escreva a conclusão aqui.
\section{Exemplo de imagem}
\vspace{0.5cm}
\begin{figure}
    \centering
    \includegraphics[width=0.5\linewidth]{USP LOGO.png}
    \caption{Caption}
    \label{fig:enter-label}
\end{figure}

\subsection{TABELA}
    {
    \begin{table}[H]
    \centering
    \begin{tabular}{|c|c|} \hline 
         A&  B\\ \hline 
         VALOR 1& 
    VALOR 2 \\ \hline\end{tabular}
    \caption{TITULO TABELA}
    \label{tab:my_label}
    \end{table}
    
    }
   


% =============================================================================
% ELEMENTOS PÓS-TEXTUAIS
% Conforme ABNT: mantém numeração sequencial
% =============================================================================
\postextual
% ----------------------------------------------------------

% -----------------------------------------------------------
% Referências bibliográficas
% ----------------------------------------------------------
\bibliography{referencias}




\end{document}
